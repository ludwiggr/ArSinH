% Diese Zeile bitte -nicht- aendern.
\documentclass[course=erap]{aspdoc}

%%%%%%%%%%%%%%%%%%%%%%%%%%%%%%%%%
%% TODO: Ersetzen Sie in den folgenden Zeilen die entsprechenden -Texte-
%% mit den richtigen Werten.
\newcommand{\theGroup}{233} % Beispiel: 42
\newcommand{\theNumber}{A316} % Beispiel: A123
\author{Ludwig Gröber \and Julian Pins \and Daniel Safyan}
\date{Sommersemester 2023} % Beispiel: Wintersemester 2019/20
%%%%%%%%%%%%%%%%%%%%%%%%%%%%%%%%%

% Diese Zeile bitte -nicht- aendern.
\title{Gruppe \theGroup{} -- Abgabe zu Aufgabe \theNumber}

\begin{document}
    \maketitle


    \section{Einleitung}
    \("\)In diesem einleitenden Abschnitt soll die Problemstellung eingeführt und beschrie- ben werden.
    Zudem muss die Aufgabenstellung analysiert und ggf. näher spezi- fiziert werden.\("\)\n

    \subsection{Einführung 1/2Seite}
    Die gestellte Aufgabe verlangt die Implementierung der Funktion f(x)=arsinh(x) im C17 Standard von C.


    Ebenso sollen gefundene Lösungen wissenschaftlich bewertet werden.
    Die Funktion area sinus hyperbolicus ist die Umkehrfunktion des sinus hyperbolicus.
    Die Gruppe der hyperbolischen Funktionen wird für die Hankel-Transformation~\cite{hankel},
    bei der Lösung bestimmter Differenzialgleichungen~\cite{differenzial}, bei der Beschreibung von Katenoiden~\cite{katenoid}
    sowie für die zeitliche Entwicklung der Ausdehnung des Universums~\cite{katenoid} benötigt.

    \subsection{Aufgabenstellung analysiert und spezifiziert 1/2Seite}
    Der area sinus hyperbolicus kann über den Logarithmus oder über ein Integral definiert werden.

    \operatorname{arsinh}(x) = \ln \left(x + \sqrt{x^2 + 1} \right) mit \, x \in R


    \operatorname{arsinh}(x) = \int\_{0}^{1} \frac{x}{\sqrt{x^2 y^2 + 1}} \,\mathrm{d}y mit \, x \in R


    Der arsinh ist im Bereich \infty < x < + \infty \, definiert.
    Im Limit kann er durch die Funktion f(x)\to \pm \ln(2|x|) angenähert werden.
    Ebenso ist als besondere mathematische Eigenschaft die Punktsymmetrie zum Ursprung (0,0) zu erwähnen.

    Intervalle werden größer an den Rändern, weil double precision nicht so genau ist.
    Relativer Fehler bleibt gleich an jeder Stelle.
    Punktsymmetrie zum Ursprung
    ar sin h = sinh^-1


    \section{Lösungsansatz}
    \("\)Der gewählte Lösungsansatz soll klar und nachvollziehbar beschrieben und analy- siert werden.
    Getroffene Entscheidungen werden diskutiert und begründet, umge- setzte Optimierungen werden erläutert.
    Wo möglich beinhaltet dies eine Gegenüber- stellung möglicher Lösungsalternativen und Überlegungen zu möglichen weiteren Optimierungen.\("\)

    Ideen: Reihenentwicklung nicht zur Runtime, sondern zur Compiletime berechnen.
    Runge Effekt: zwischen Integralgrenzen gut angenähert, wenn mit Polynom genähert wird. -> Deshalb funktioniert die Reihenentwicklung nicht.
    Lösung: Splines über großen Bereich mit x^3 Polynom annähern. Intervall gleich verteilt. alpha x^3 + beta x^2 + gamma x + delta
    Annäherung für große Werte: x^2 + 1 = x^2
    Ableitung per Tailor-Reihe
    Idee: 1. Chilliger Lookup Table mit Interpolation 2. Splines Lookup Table
    Entwicklung als echte Reihendarstellung muss begrenzt werden im Wertebereich.

    Grunstätzlich möglichst wenig zur Runtime berechnen.
    Reihe: Basisfunktionen + Lineares Gleichungssystem
    Lagrange Polynome
    Tailor-Reihe / Tailor Entweicklung

    \subsection{Naive Implementierung Reihenentwicklung ]-1/1[}

    \subsection{Naive Implementierung Tabellen-Lookup ]-Inf/+Inf[}

    \subsection{Vergleich der beiden Ansätze}

    \subsection{Optimierte Implementierung Reihenentwicklung ]-1/1[}
    Complexe Instruktionen, Wurzel, Log, Exponent, SIMD?

    \subsection{Optimierte Implementierung Tabellen-Lookup ]-Inf/+Inf[}
    Idee: Hash-Map

    \subsection{Implementierung mit komplexen Instruktionen}


    \section{Genauigkeit}
    \("\)Die Korrektheit bzw. Genauigkeit (jeweils wo angemessen) der Implementierung und des Ansatzes
    soll gezeigt und bewertet werden.
    Nach Möglichkeit sind hierfür auch automatisierte Tests zu implementieren und abzugeben.
    Außerdem sollen sinnvolle und repräsentative Beispiele für Eingaben und die dazu berechneten Ergebnisse gezeigt werden.\("\)

    Im gegebenen Kontext ist die Genauigkeit der Lösung als die Abweichung vom funktionswert der mathematisch definierten Funktion zu verstehen.


    \section{Performanzanalyse}
    \("\)In diesem Abschnitt soll die Performanz der Implementierung analysiert und bewertet werden.
    Hierzu sind geeignete Methoden zu wählen, beispielsweise Zeitmessungen.
    Bei Zeitmessungen ist mindestens eine weitere zur Evaluierung der Implementierung geeignete Vergleichsimplementierung
    heranzuziehen und diese auf verschiedenen repräsentativen Eingaben auszuführen.
    Die Wahl der Eingaben ist zu begründen und die Messumgebung und -methodik sind genau zu spezifizieren.
    Die gewonnenen Ergebnisse sind zu bewerten und Ursachen für beobachtete Performanzunterschiede zu benennen.\("\)

    \subsection{Methodik und Annahmen 0,5Seite}

    \subsubsection{Zeitmesung der naiven Reihenentwicklung}

    \subsubsection{Zeitmesung des naiven Tabellen-Lookup}

    \subsubsection{Zeitmesung der optimierten Reihenentwicklung}
    \("\)ziehen Sie als weiteren Vergleich eine C-Implementierung unter Nutzung von komplexeren Instruktionen, die beispielsweise eine Wurzelberechnung durchführen, heran.\("\)

    \subsubsection{Zeitmesung des optimierten Tabellen-Lookup}
    \("\)ziehen Sie als weiteren Vergleich eine C-Implementierung unter Nutzung von komplexeren Instruktionen, die beispielsweise eine Wurzelberechnung durchführen, heran.\("\)

    \subsection{Bewertung, Einordnung und Erklärung der Ergebnisse}


    \section{Zusammenfassung und Ausblick}
    \("\)Neben einer kurzen Zusammenfassung des umgesetzten Projektes sollte dieser Abschnitt einen kurzen Ausblick enthalten.
    In diesem wird beurteilt, wo zusätzliches Potential für Verbesserungen existiert, ob rückblickend eine andere
    Lösungsalternative besser wäre.\("\)

    Provide a summary of your work and discuss future prospects~\cite{intel2017man}.


% TODO: Fuegen Sie Ihre Quellen der Datei Ausarbeitung.bib hinzu
% Referenzieren Sie diese dann mit \cite{}.
% Beispiel: CR2 ist ein Register der x86-Architektur~\cite{intel2017man}.
    \bibliographystyle{plain}
    \bibliography{Ausarbeitung}{}
% Sämtliche in der Ausarbeitung verwendeten Quellen sind hier aufzuführen.
% Es sollen nur zitierfähige Quellen verwendet werden. Wir empfehlen die Verwendung von BibTEX.

\end{document}
